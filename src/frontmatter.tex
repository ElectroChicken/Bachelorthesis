\pdfbookmark[0]{Titlepage}{titlepage}
\titlehead{
    \begin{center}
        \includegraphics[width=3cm]{gfx/unilogo} \\ \medskip
        \myInstitute \\ 
        \myGroup
    \end{center}
}

\title{\myTitle}
\subtitle{\mySubtitle}

\author{\myName}
\date{\myDate}

\publishers{First supervisor: \myFirstSupervisor \\
            Second supervisor: \mySecondSupervisor
}

\maketitle

\cleardoublepage

%%%%%%%%%%%%%%%%%%%%%%%%%%%%%%%%%%%%%%%%%%%%%%%%%%%%%%%%%%%%%%%%%%%%%

\begingroup
    \let\cleardoublepage\relax

    \pdfbookmark[0]{Abstract}{abstractEN}
    \addchap*{Abstract}
    Answer Set Programming (ASP) is a popular declarative programming paradigm. However, relatively little work has been done on improving the development process for answer set programs. Workflow tools such as a testing suite are an integral part of the software development process for other programming paradigms but are almost nonexistent for ASP. The concept of code coverage is widely used during testing to evaluate or generate new test cases. 
    
    In this thesis, we use the existing definitions of five code coverage metrics for propositional normal logic programs and develop a practical way to compute them using the power of ASP itself. We also investigate ideas how these metrics can be extended to also be applicable to more complex program classes. Specifically, we define syntactic transformations of answer set programs which allow us to compute all five coverage metrics for multiple test cases all at once. We also implement these transformations in a prototype coverage testing tool. Finally, we show that the chosen approach to checking coverage for propositional programs can be easily adapted to also work with further program classes.
    
    
    %- ASP good but very little workflow tools to help development
    %- Code coverage widely used concept in other programming paradigms
    %- Objective: develop a way to allow computation of coverage metrics by Jan et al using  the power of ASP
    %- Investigate ideas how to extend the metrics for further program classes
 
    %- Use a syntactic transformation for each metric to compute coverage for multiple test inputs at once
    %- Develop prototype tool
 
    %- The power of ASP does most of the work to compute coverage metrics using our approach
    %- Only few adjustments needed to make other program classes work
 
    %- We laid ground work for practical coverage testing of ASP
    %- More research is necessary especially for complex programs 
    
    \clearpage

    \begin{otherlanguage}{ngerman}
        \pdfbookmark[0]{Zusammenfassung}{abstractDE}
        \addchap*{Zusammenfassung}
        Answer Set Programming (ASP) ist ein beliebtes deklaratives Programmierparadigma. Allerdings war die Verbesserung des Entwicklungsprozesses von Answer Set Programmen bisher nur selten Thema der Forschung. Workflow-Tools wie eine Testsuite sind ein wichtiger Teil des Softwareentwicklungsprozesses bei anderen Programmierparadigmen, existieren aber kaum für ASP. Das Konzept der Testabdeckung ist weit verbreitet beim Testen um vorhandene Testfälle zu evaluieren oder neue zu generieren.

        In dieser Arbeit benutzen wir die bereits existierenden Definitionen von fünf Testabdeckungsmetriken für propositionale Logikprogramme und entwickeln einen praktischen Weg diese mit Hilfe von ASP zu berechnen. Wir untersuchen zudem Ansätze, diese Metriken so zu erweitern, dass sie auch auf komplexere Programmklassen angewendet werden können. Konkret definieren wir syntaktische Transformationen von Answer Set Programmen, die es uns erlauben, alle fünf Coverage Metriken für beliebig viele Testfälle gleichzeitig zu berechnen. Außerdem implementieren wir diese Transformationen in dem Prototyp eines Coverage-Test-Tools. Schließlich zeigen wir, dass der gewählte Ansatz um Code Coverage für propositionale Programme zu testen so angepasst werden kann, dass er auch bei komplexeren Programmklassen eingesetzt werden kann.
    \end{otherlanguage}

    \vfill
\endgroup

\cleardoublepage

%%%%%%%%%%%%%%%%%%%%%%%%%%%%%%%%%%%%%%%%%%%%%%%%%%%%%%%%%%%%%%%%%%%%%

%\pdfbookmark[0]{Acknowledgments}{acknowledgments}
%\chapter*{Acknowledgments}
%Danksagung!

%\cleardoublepage

\pdfbookmark[0]{\contentsname}{toc}
\tableofcontents

\cleardoublepage

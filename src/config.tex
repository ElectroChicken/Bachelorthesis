\usepackage[ngerman,english]{babel}

\usepackage[T1]{fontenc}

\usepackage{lmodern}

\usepackage{scrhack}

\usepackage{csquotes}

\usepackage[printonlyused]{acronym}         % für Liste der Abkürzungen

\usepackage{graphicx}                       % für schöne Graphiken

\usepackage{tabularx}                       % für schöne Tabellen

\usepackage{listings}                       % für schöne code Beispiele
\lstset{
    basicstyle      = \ttfamily\small,
    frame           = tb,
    numbers         = left,
    numbersep       = 8pt,
    numberstyle     = \ttfamily\scriptsize,
    breaklines      = true,
    captionpos      = b,
    abovecaptionskip    = 0.5\baselineskip,
}

\usepackage[
    style           = alphabetic,
    backref         = true,
    backrefstyle    = two,
]{biblatex}                                     % Biblatex setup

\bibliography{bibliography.bib}                 % Pfad zu Bibliographie

\usepackage{amsmath}                            % Matehmatische Symbole
\DeclareMathOperator{\Def}{Def_P}               % eigene (oft benutzte) Operatoren
\DeclareMathOperator{\SuppR}{SuppR}
\DeclareMathOperator{\AS}{AS}
\DeclareMathOperator{\inp}{inp}
\usepackage{amssymb}

\usepackage{amsthm}
\theoremstyle{definition}
\newtheorem{definition}{Definition}[chapter]
\newtheorem{example}{Example}[chapter]
\newtheorem{theorem}{Theorem}[chapter]
%%% add line below chapter title
\renewcommand*\chapterheadendvskip{\noindent\rule{\linewidth}{1pt}\par\vspace{\baselineskip}}

%\usepackage{xcolor}
%\definecolor{CTtitle}{cmyk}{1, 0.50, 0, 0}

\newcommand{\myTitle}{Computing coverage metrics for answer set programms}
\newcommand{\mySubtitle}{Bachelorthesis}
\newcommand{\myName}{Jakob Westphal}
\newcommand{\myFirstSupervisor}{Prof. Dr. Torsten Schaub}
\newcommand{\mySecondSupervisor}{Tobias Stolzmann}
\newcommand{\myUni}{University of Potsdam}
\newcommand{\myGroup}{Knowledge Processing and Information Systems}
\newcommand{\myInstitute}{Institute of Computer Science}
\newcommand{\myLocation}{Berlin}
\newcommand{\myDate}{29. April 2023}
\newcommand{\myVersion}{1.0.0}

\newcommand{\code}[1]{\texttt{\small #1}}       % für inline code

\usepackage{hyperref}                           % für Verlinkungen im Dokument
\hypersetup{
    pdfauthor       = {\myName},
    pdftitle        = {\myTitle},
    pdfstartview    = FitV,
    pdfpagemode     = UseNone,
}

\usepackage[noabbrev]{cleveref}                  % schönere Verlinkungen

\usepackage{comment}                             % multiline comments

\usepackage{tikz}                                % Graphen malen
\usetikzlibrary {positioning}

\chapter{Introduction}
\label{ch:Introduction}

Einleitung
- what am i doing? (Fragestellung/Goal)

- why am i doing it?

\section{Coverage in general}
\label{sec:Introduction/Coverage in general}
- theoretical discussion of coverage metrics in this paper

%%%%%%%%%%%%%%%%%%%%%%%%%%%%%%%%%%%%%%%%%%%%%%%%%%%%%%%%%%%%%%%%%

\chapter{Test}
\label{ch:Test}

\begin{enumerate}
    \item Bla.\\
    (Bla.)
    \item Bla. (Bla.)
\end{enumerate}

\Cref{ch:Introduction} Referenz zu Kapitel.
\Cref{sec:Test/Perzepte_und_Symbole} Referenz zu Unterkapitel. 
\Cref{fig:cups_yolo} Referenz zu Bild.
\Cref{lst:cups_symbolic} Referenz zu Listing
\textcite{Jan+10} Zitat mit Namen der Autoren
\cite{Jan+10} Zitat nur mit Abkürzung
\ac{ASP} Link zu Abkürzungen
\emph{symbol grounding problem} 
\marginpar{Symbol} Randkommentar
\footnote{Bla} Fussnote

\section{Perzepte und Symbole}
\label{sec:Test/Perzepte_und_Symbole}
Bla.

%\begin{definition}
%    \label{def:Perzept}
%    Ein \emph{Perzept}\marginpar{Perzept} ist der sensorische Eindruck eines physikalischen Objektes zu einem bestimmten Zeitpunkt.
%\end{definition}

\begin{figure}
    \centering
    \includegraphics[width=0.4\textwidth]{gfx/unilogo.jpg}
    \caption{Ein Kamerabild mit eingezeichneten Perzepten.}
    \label{fig:cups_yolo}
\end{figure}


\begin{lstlisting}[float,caption={Eine symbolische Beschreibung der Objekte in bla.},label=lst:cups_symbolic]
symbol(cup_1; cup_2; cup_3; spoon; diningtable).

is_on(
    cup_1, diningtable;
    cup_2, diningtable;
    cup_3, diningtable
).

is_inside_of(spoon, cup_3).

contains(
    cup_1, coffee;
    cup_2, coffee;
    cup_3, hot_chocolate
).

\end{lstlisting}



%\begin{figure}
%    \centering
%    \begin{tikzpicture}[
%        ->,
%        >={Stealth[round]},
%        align=center,
%        state/.style={
%            draw,
%            rectangle,
%            rounded corners=3mm
%        },
%        every edge/.append style={thick}
%    ]
%        \node (A) [state]             {Symbol verankert};
%        \node (B) [state, below=of A] {Symbol nicht verankert};
%
%        \node (1) [left =25mm of A, font=\scriptsize] {ausgehend von\\einem Perzept};
%        \node (2) [right=25mm of A, font=\scriptsize] {ausgehend von\\einem Symbol};
%        \node (3) [below=of B,      font=\scriptsize] {Symbol gelöscht};
%
%        \path (A) edge [loop above] node [above] {\emph{Verfolgen}}    (A);
%        \path (A) edge [bend left]  node [right] {\emph{Verlieren}}    (B);
%        \path (B) edge [bend left]  node [left]  {\emph{Wiederfinden}} (A);
%        \path (1) edge              node [above] {\emph{Entdecken}}    (A);
%        \path (2) edge              node [above] {\emph{Finden}}       (A);
%        \path (B) edge              node [right] {\emph{Zerstören}}    (3);
%    \end{tikzpicture}
%    \caption{Die Verankerungsfunktionen als Zustandsübergänge, frei nach \cite[Abbildung~4]{Gün+18}.}
%    \label{fig:anchoring_functions_as_state_transitions}
%\end{figure}




$$\operatorname{match}(\sigma, \gamma) \Leftrightarrow \forall p \in \sigma \; \exists \phi \in \operatorname{feat}(\gamma): \; g(p, \phi, \gamma(\phi))$$


\begin{table}
    \centering
        \begin{tabularx}{0.5\textwidth}{l|X|l}
            $M$             & $P_1^M$                               & $Cn(P_1^M)$  \\
            \hline
            $\emptyset$     & $\{ a \leftarrow a, b \leftarrow \}$  & $\{ b \}$     \\
            $\{ a \}$       & $\{ a \leftarrow a \}$                & $\emptyset$   \\
            $\{ b \}$       & $\{ a \leftarrow a, b \leftarrow \}$  & $\{ b \}$     \\
            $\{ a, b \}$    & $\{ a \leftarrow a \}$                & $\emptyset$   \\
        \end{tabularx}
    \caption[$P_1 = \{ a \leftarrow a, b \leftarrow naf a \}$ hat ein stabiles Modell.]{$P_1 = \{ a \leftarrow a, b \leftarrow naf a \}$ hat ein stabiles Modell $\{ b \}$.}
    \label{tab:Ein_stabiles_Modell}
\end{table}

%\begin{example}
%    Das Programm $P = \{ \; \{ a, b\} \; \}$ hat vier stabile Modelle, nämlich die Elemente von $2^{\{a, b\}}$.
%\end{example}

%\begin{example}
%   Das Programm
%    $$
%        P =
%        \begin{Bmatrix}
%            \operatorname{cup}(1) \\
%            \operatorname{cup}(2) \\
%            1~\{~\operatorname{blue}(X) : \operatorname{cup}(X)~\}~1 \\
%        \end{Bmatrix}
%    $$
%    hat die Grundinstanz
%    $$
%        \grd(P) =
%        \begin{Bmatrix}
%            \operatorname{cup}(1) \\
%            \operatorname{cup}(2) \\
%            1~\{~\operatorname{blue}(1),~\operatorname{blue}(2)~\}~1 \\
%        \end{Bmatrix}
%    $$
%    und die stabilen Modelle $\{\operatorname{cup}(1),{ }\operatorname{cup}(2),{ }\operatorname{blue}(1)\}$ und \linebreak$\{\operatorname{cup}(1),{ }\operatorname{cup}(2),{ }\operatorname{blue}(2)\}$.
%\end{example}

\begin{proof}
    Zu jeder Teilmenge $M \subseteq A = \{ a, b, c \}$ ist $P^M = P$.
    Die Teilmengen~$\emptyset$, $\{ a \}$, $\{ c \}$, $\{ a, b \}$ und $\{ b, c \}$ sind keine Modelle von $P^M$. $\{ a, c \}$, $\{ a, b, c\}$ und $\{ b \}$ sind Modelle von $P^M$.
    $\{ a, b, c\}$ ist kein minimales Modell von $P^M$, da $\{ b \} \subseteq \{ a, b, c\}$.
    Da $\{ a, c \} \nsubseteq \{ b \}$ und $\{ b \} \nsubseteq \{ a, c \}$, sind beide Modelle minimal und damit stabile Modelle von $P$.
\end{proof}

\code{\#show p(X,Y) : q(X).}

Test für \code{\#show p(X)} in einer Zeile.
%\lstinputlisting[float,caption={[Ein Graph mit 6 Knoten und 17 Kanten\\(\code{graph.lp}).]Ein Graph mit 6 Knoten und 17 Kanten (\code{graph.lp}).},label=lst:graphcolor/graph.lp]{../../code/graphcolor/graph.lp}

\begin{align*}
    &X                          &=\ &\{ \text{cup}_1, \text{cup}_2 \} \\
    &\Pi                        &=\ &\{ \pi_1, \pi_2, \pi_3 \} \\
    &\Phi                       &=\ &\{ \text{coffee}, \text{tea}, \text{hot}, \text{cold} \} \\
    &T                          &=\ &\{ t_1, t_2 \} \\
    &\beta(\text{cup}_1, t_1)   &=\ &\{ \text{coffee} \} \\
    &\beta(\text{cup}_2, t_1)   &=\ &\emptyset \\
    &\beta(\pi_1, t_1)          &=\ &\{ \text{coffee} \} \\
    &\beta(\pi_2, t_1)          &=\ &\{ \text{tea}, \text{cold} \} \\
    &\beta(\pi_3, t_2)          &=\ &\{ \text{tea} \}
\end{align*}


\cleardoublepage
